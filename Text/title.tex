\newpage
\thispagestyle{empty}
\begin{adjustwidth}[]{0cm}{0cm}
\begin{center}
\begin{linespread}{1}


\small{
МИНИСТЕРСТВО ОБРАЗОВАНИЯ И НАУКИ РОССИЙСКОЙ ФЕДЕРАЦИИ\\
Федеральное государственное бюджетное образовательное учреждение\\
высшего образования\\
\textbf{<<Южно-Уральский государственный университет\\
(национальный исследовательский университет)>>\\
Высшая школа электроники и компьютерных наук\\
Кафедра системного программирования}
}

\vspace{\stretch{1}}


{
\large\textbf{Разработка системы для поиска припева в тексте песни}
}

\vspace{2em}

КУРСОВАЯ РАБОТА \\
по дисциплине «Программная инженерия»\\
ЮУрГУ – 09.03.04.2023.308-059.КР


\vspace{\stretch{1}}


\parbox[t]{7cm}{
Нормоконтролер,\\
профессор кафедры СП, д.ф-м.н., \\  
доцент\\[0.5em]
\underline{\hspace{2.5cm}} М.Л.~Цымблер \\[0.5em]
``\underline{\qquad}''\underline{\hspace{2.5cm}} 2024~г.
}
\hfill{}
\parbox[t]{7cm}{
Научный руководитель: \\
профессор кафедры СП, \\
д.ф.-м.н., доцент\\[0.5em]
\underfield{} М.Л.~Цымблер \\[2.5em]
Автор работы, \\
студент группы КЭ-303\\[0.5em]
\underfield{} А.А.~Летуновский \\[2.5em]
Работа защищена \\
с оценкой: \underfield{} \\[0.5em]
``\underline{\qquad}''\underfield{} 2024~г.
}

\vspace{\stretch{1}}

Челябинск 2024

\end{linespread}
\end{center}
\end{adjustwidth}

\pagebreak
