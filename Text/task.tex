\newpage
\thispagestyle{empty}

\begin{adjustwidth}{-1.5cm}{0.5cm}
\begin{linespread}{1}
\begin{center}


\small{
МИНИСТЕРСТВО ОБРАЗОВАНИЯ И НАУКИ РОССИЙСКОЙ ФЕДЕРАЦИИ\\
Федеральное государственное бюджетное образовательное учреждение\\
высшего образования\\
\textbf{<<Южно-Уральский государственный университет\\
(национальный исследовательский университет)>>\\
Высшая школа электроники и компьютерных наук\\
Кафедра системного программирования}
}



\vspace{2em}

\hfill{}
\parbox{7cm}{
УТВЕРЖДАЮ \\
Зав. кафедрой СП \\[0.5em]
\underfield{} Л.Б.~Соколинский \\[0.5em]
"\underline{\qquad}"\underfield{}2024
}

\vspace{2em}

\textbf{ЗАДАНИЕ} \\
% \parbox[t]{14cm}{
\textbf{на выполнение выпускной курсовой работы}\\
студенту группы КЭ-303\\
Летуновскому Арсению Александровичу,\\
обучающемуся по направлению 09.03.04 «Программная инженерия» 
% }

\end{center}

\vspace{2em}

{
\small
\begin{enumerate}
	\bf\item Тема работы \rm
	(утверждена приказом ректора от  \No~)\\
	Разработка системы для поиска припева в тексте песни

	\bf\item Срок сдачи студентом законченной работы: \rm
	31.05.2024~г.

	\bf\item Исходные данные к работе\rm
	\begin{enumerate}%[leftmargin=0.35cm]
		\raggedright

		\item Imani, S., Madrid, F., Ding, W. et al. Introducing time series snippets: a new primitive for summarizing long time series // Data Min Knowl Disc 34, 2020. --P. 1713–-1743.

        \item Watanabe K., Goto M. A Chorus-Section Detection Method for Lyrics Text. // Proceedings of the 21th International Society for Music Information Retrieval Conference, {ISMIR} 2020, Montreal, Canada, October 11--16, 2020. --P 351–359

	\end{enumerate}

	\bf\item Перечень подлежащих разработке вопросов\rm
	\begin{enumerate}
		\item Выполнить анализ предметной области и провести обзор существующих решений.
		\item Выполнить разработку алгоритма поиска припева в тексте песни на основе поиска типичных подпоследовательностей временного ряда. 
		\item Разработать приложение для использования алгоритма поиска припева в тексте песни.
		\item Разработать тестовые наборы и провести тестирование разработанного приложения.
        \item Оценить точность полученных результатов относительно истинной разметки.
	\end{enumerate}

	\bf\item Дата выдачи задания: \rm
	"\underline{\qquad}"\underfield{}2024~г.
\end{enumerate}

\vspace{1em}

\noindent
\textbf{Научный руководитель}
\hfill
\hbox to 8em{М.Л.~Цымблер\hfill}

\vspace{1em}

\noindent
\textbf{Задание принял к исполнению}
\hfill
\hbox to 8em{A.А.~Летуновский\hfill}

}

\thispagestyle{empty}

\end{linespread}
\end{adjustwidth}

\pagebreak
