\newpage
\sectionnonumber{Введение}
\subsection*{Актуальность темы}
Описание причин создания данного проекта.
\subsection*{Цель и задачи исследования}
Целью данной курсовой работы является разработка системы поиска припева в тексте песни.

Для достижения поставленной цели необходимо решить следующие задачи:
\begin{enumerateparen}
    \item Провести обзор аналогичных проектов.   
    \item Выполнить разработку алгоритма классификации припевов и куплетов.
    \item Разработать приложения для разметки текстов песен, настройки и запуска алгоритмов, анализа результатов и визуализации данных.
    \item Провести разметку песен и составить из них набор данных.
    \item Провести тестирование разработанного алгоритма.
\end{enumerateparen}
\subsection*{Структура и объем работы}
\newcounter{biconvert}
\setcounter{biconvert}{\totvalue{bibitems}}
\regtotcounter{chapter}
Курсовая работа состоит из введения, четырех глав, заключения и библиографического списка. Объем работы составляет \numplural{\getpagerefnumber{LastPage}}{страниц}{страницу}{страницы}, объем списка литературы~-- \numplural{\arabic{biconvert}}{наименований}{наименование}{наименования}.
\subsection*{Содержание работы}
Первая глава, <<АНАЛИЗ ПРЕДМЕТНОЙ ОБЛАСТИ>>, содержит обзор предметной области, а так же анализ существующих решений по тематике курсовой работы.

Во второй главе, <<ПРОЕКТИРОВАНИЕ>>, содержатся функциональные и нефункциональные 
требования к разрабатываемому приложению. Приведены варианты использования программы, архитектура приложения, его компоненты и пользовательский интерфейс.

Третья глава, <<РЕАЛИЗАЦИЯ>>, содержит описание программных средств, используемых в процессе реализации системы, описание реализации алгоритма поиска припева песни, а также пользовательского интерфейса.

В четвертой главе, <<ТЕСТИРОВАНИЕ>>, представлено функциональное тестирование и оценка точности полученных результатов относительно истинной разметки.

В заключении подведены основные итоги выполненной работы.