\newpage
\section{Анализ предметной области}
\label{sec:Background}

\subsection{Описание предметной области}
Обзор временных рядов, сниппетов временных рядов, а также алгоритмов поиска данных сниппетов во временном ряде.
\vspace{2em}
\subsection{Анализ аналогичных проектов}
Работ со схожей тематикой немного. Наиболее близким аналогом является работа японских исследователей\cite{WatanabeG20}, в которой для выделения из текста песни куплетов и припевов используется модель, основанная на обученной нейронной сети. Данная нейронная сеть анализирует девять матриц самоподобия, составленных на основе текста песни.

После того как были созданы матрицы самоподобия, высчитываются векторы признаков с помощью сверточной нейронной сети. Данные векторы используются двунаправленными сетями с длительной кратковременной памятью для разметки текста песни.

Еще одним способом выделения куплетов и припевов является анализ звуковых дорожек песен. Данный способ является более исследованным, чем анализ текста.

Например, модель "DeepChorus" \cite{DeepChorus}, которая использует сочетание многомасштабной сверточной сети и self-attention сверточной нейронной сети. После обработки сигнала этими сетями выходной вектор проходит процесс бинаризации, где происходит разделение данных на припевы и куплеты.



